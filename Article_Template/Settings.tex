% Font and spacing
%#######################################################################

%Scalable font similar to cm.
\usepackage{lmodern} 

%Sans serif font.
%\renewcommand{\familydefault}{\sfdefault} 

%1.5 line spacing and fix for arrays and tables.
%\usepackage[onehalfspacing]{setspace}
%\setdisplayskipstretch{}
%\renewcommand{\arraystretch}{0.75}






% Alter some LaTeX defaults for better treatment of figures:
%#######################################################################

% General parameters, for all pages:
\renewcommand{\topfraction}{0.9}	% max fraction of floats at top
\renewcommand{\bottomfraction}{0.8}	% max fraction of floats at bottom

%   Parameters for text pages:
\setcounter{topnumber}{2}
\setcounter{bottomnumber}{2}
\setcounter{totalnumber}{4}     % 2 may work better
\setcounter{dbltopnumber}{2}    % for 2-column pages
\renewcommand{\dbltopfraction}{0.9}	% fit big float above 2-col. text
\renewcommand{\textfraction}{0.07}	% allow minimal text w. figs

%   Parameters for float pages (floatpagefraction must be less than topfraction):
\renewcommand{\floatpagefraction}{0.7}	% require fuller float pages
\renewcommand{\dblfloatpagefraction}{0.7}	% require fuller float pages








% Language
%#######################################################################

\usepackage[english]{babel} 







% Hyperlinks
%#######################################################################

\usepackage[colorlinks=true, linkcolor=Cerulean!75!black, citecolor=green!60!black, urlcolor=Cerulean!75!black, linktocpage=true]{hyperref} 










% Mathematics options
%#######################################################################


%Page breakes in multiline formulas.
\allowdisplaybreaks 

%Bold mathematics in headings, etc..
\makeatletter
\g@addto@macro\bfseries{\boldmath}
\makeatother



%Definition of the usual theorem environments.
\newtheoremstyle{theorem}{\topsep}{\topsep}{\itshape}{}{\bf}{.\ \\}{.5ex}{}
\newtheoremstyle{definition}{\topsep}{\topsep}{}{}{\bf}{.\ \\}{.5ex}{}
\newtheoremstyle{info}{\topsep}{\topsep}{}{}{\bfseries \normalsize }{.\ \\}{.5ex}{}


\theoremstyle{theorem}
\newtheorem{theoremEnvironment}{Theorem}[section]
\newtheorem{corollaryEnvironment}[theoremEnvironment]{Corollary}
\newtheorem{lemmaEnvironment}[theoremEnvironment]{Lemma}
\newtheorem{propositionEnvironment}[theoremEnvironment]{Proposition}

%Variable theorem name using \renewcommand{\varName}{new name}.
\newcommand{\varName}{varTheorem} 
\newtheorem{varTheorem}[theoremEnvironment]{\varName}
\newtheorem{varLemma}[theoremEnvironment]{\varName}
\newtheorem{varCorollary}[theoremEnvironment]{\varName}

\theoremstyle{definition}
\newtheorem{definitionEnvironment}[theoremEnvironment]{Definition}
\newtheorem{proofEnvironment}[theoremEnvironment]{Proof}
\newtheorem{varDefinition}[theoremEnvironment]{\varName}

\theoremstyle{info}
\newtheorem{remarkEnvironment}[theoremEnvironment]{Remark}
\newtheorem{exampleEnvironment}[theoremEnvironment]{Example}







% convenient combination of environments and frames:

\newenvironment{theorem}[1][]{\begin{theorem_frame}\begin{theoremEnvironment}[#1]}{\end{theoremEnvironment}\end{theorem_frame}}
\newenvironment{proposition}[1][]{\begin{theorem_frame}\begin{theoremEnvironment}[#1]}{\end{theoremEnvironment}\end{theorem_frame}}
\newenvironment{lemma}[1][]{\begin{lemma_frame}\begin{lemmaEnvironment}[#1]}{\end{lemmaEnvironment}\end{lemma_frame}}
\newenvironment{corollary}[1][]{\begin{corollary_frame}\begin{corollaryEnvironment}[#1]}{\end{corollaryEnvironment}\end{corollary_frame}}
\newenvironment{definition}[1][]{\begin{definition_frame}\begin{definitionEnvironment}[#1]}{\end{definitionEnvironment}\end{definition_frame}}
\newenvironment{remark}[1][]{\begin{remark_frame}\begin{remarkEnvironment}[#1]}{\end{remarkEnvironment}\end{remark_frame}}
\newenvironment{example}[1][]{\begin{example_frame}\begin{exampleEnvironment}[#1]}{\end{exampleEnvironment}\end{example_frame}}
\renewenvironment{proof}[1][]{\begin{proof_frame}\begin{proofEnvironment}[#1]}{\end{proofEnvironment}\end{proof_frame}}

\newenvironment{vartheorem}[1][]{\begin{theorem_frame}\begin{varTheorem}[#1]}{\end{varTheorem}\end{theorem_frame}}
\newenvironment{varlemma}[1][]{\begin{lemma_frame}\begin{varLemma}[#1]}{\end{varLemma}\end{lemma_frame}}
\newenvironment{varcorollary}[1][]{\begin{corollary_frame}\begin{varCorollary}[#1]}{\end{varCorollary}\end{corollary_frame}}
\newenvironment{vardefinition}[1][]{\begin{definition_frame}\begin{varDefinition}[#1]}{\end{varDefinition}\end{definition_frame}}




% Convenient notation for quantum mechanics
%#######################################################################



%Notation quantum mechanics.
\newcommand{\bra}[1]{\langle #1 |}
\newcommand{\ket}[1]{| #1 \rangle}
\newcommand{\braket}[1]{\langle #1 \rangle}
\newcommand{\Bra}[1]{\left\langle #1 \right|}
\newcommand{\Ket}[1]{\left| #1 \right\rangle}
\newcommand{\Braket}[1]{\left\langle #1 \right\rangle}

%\boxed with additional horizontal space.
\newcommand{\widebox}[1]{\boxed{\hspace{1em}#1\hspace{1em}}}




% Additional packages/settings
%#######################################################################


\usepackage{tikz-3dplot}




